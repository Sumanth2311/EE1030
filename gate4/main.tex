%iffalse
\let\negmedspace\undefined
\let\negthickspace\undefined
\documentclass[journal,12pt,onecolumn]{IEEEtran}
\usepackage{cite}
\usepackage{amsmath,amssymb,amsfonts,amsthm}
\usepackage{algorithmic}
\usepackage{multicol}
\usepackage{circuitikz}
\usepackage{tikz}
\usepackage{graphicx}
\usepackage{textcomp}
\usepackage{xcolor}
\usepackage{txfonts}
\usepackage{listings}
\usepackage{enumitem}
\usepackage{mathtools}
\usepackage{gensymb}
\usepackage{comment}
\usepackage[breaklinks=true]{hyperref}
\usepackage{tkz-euclide} 
\usepackage{listings}
\usepackage{gvv} 
\usepackage{tikz}
\usetikzlibrary{shapes,arrows} 

%\def\inputGnumericTable{}                                
\usepackage[latin1]{inputenc}                                
\usepackage{color}                                            
\usepackage{array}                                            
\usepackage{longtable}                                       
\usepackage{calc}                                             
\usepackage{multirow}                                         
\usepackage{hhline}                                           
\usepackage{ifthen}                                           
\usepackage{lscape}
\usepackage{tabularx}
\usepackage{array}
\usepackage{float}
\newtheorem{theorem}{Theorem}[section]
\newtheorem{problem}{Problem}
\newtheorem{proposition}{Proposition}[section]
\newtheorem{lemma}{Lemma}[section]
\newtheorem{corollary}[theorem]{Corollary}
\newtheorem{example}{Example}[section]
\newtheorem{definition}[problem]{Definition}
\newcommand{\BEQA}{\begin{eqnarray}}
\newcommand{\EEQA}{\end{eqnarray}}
\newcommand{\define}{\stackrel{\triangle}{=}}
\theoremstyle{remark}


% Marks the beginning of the document
\begin{document}
\bibliographystyle{IEEEtran}
\vspace{3cm}

\title{\textbf{2018-EE-53-65}}
\author{AI24BTEC11027 - R Sumanth}
\maketitle
\bigskip

\renewcommand{\thefigure}{\theenumi}
\renewcommand{\thetable}{\theenumi}
\setlength{\columnsep}{2.5em}
\begin{enumerate}

\item Let $f(x)=3X^3-7x^2+5x+6.$ The maximum value of $f(x)$ over the interval $\sbrak{0,2}$ is $\underline{\hspace{2cm}}$ (up to $1$ decimal place).\\

\item Let $A=\myvec{1 & 0 & -1\\ -1 & 2 & 0\\ 0 & 0 & -2}$ and $B=A^3-A^2-4A+5I$, where I is the $3\times3$ identity determinant of $B$ is $\underline{\hspace{2cm}}$ (up to $1$ decimal place).\\

\item The capacitance of an air-filled parallel-plate capacitor is $60 pF.$ When a dielectric slab whose thickness is half the distance between the plates, is placed on one of the plates covering it entirely, the capacitance becomes $86 pF.$ Neglecting the fringing effects, the relative permittivity of the dielectric is $\underline{\hspace{2cm}}$  (up to $2$ decimal place).\\ 

\item The unit step response $y(t)$ of a unity feedback system with open loop transfer function $G(s)H(s)=\frac{K}{(s+1)^2(s+2)}$ is shown in the figure. The value of $k$ is $\underline{\hspace{2cm}}$  (up to $2$ decimal place).

\begin{figure}[!ht]
\centering
\resizebox{0.5\textwidth}{!}{%
\begin{circuitikz}
\tikzstyle{every node}=[font=\Large]
\draw [->, >=Stealth] (-5,8.75) -- (-3.75,8.75);
\draw [->, >=Stealth] (-5,8.25) -- (-3.75,8.25);
\draw [->, >=Stealth] (-5,7.75) -- (-3.75,7.75);
\draw [->, >=Stealth] (-5,7.25) -- (-3.75,7.25);
\draw [->, >=Stealth] (-5,6.75) -- (-3.75,6.75);
\draw  (-5,8.75) rectangle (-3.75,6.25);
\draw [->, >=Stealth] (-5,6.25) -- (-3.75,6.25);
\draw [->, >=Stealth] (0,8.75) -- (1.25,8.75);
\draw [->, >=Stealth] (0,8.25) -- (1.25,8.25);
\draw [dashed] (-5,6.25) .. controls (-2.75,8.25) and (-1.75,9) .. (2.5,8.75);
\draw [short] (0,6.25) -- (0,8.75);
\draw [->, >=Stealth] (0,6.25) -- (0.25,6.25);
\draw [short] (1.25,8.75) .. controls (1.25,7.5) and (0.75,6.25) .. (0,6.25);
\draw [->, >=Stealth] (0,7.75) -- (1.25,7.75);
\draw [->, >=Stealth] (0,7.25) -- (1,7.25);
\draw [->, >=Stealth] (0,6.75) -- (0.75,6.75);
\draw  (-5,6.25) rectangle (2,6);
\draw [short] (2.25,6.25) -- (2.5,6.25);
\draw [<->, >=Stealth] (2.5,8.75) -- (2.5,6.25);
\draw [->, >=Stealth] (-6.75,6.25) -- (-6.75,7.25);
\draw [->, >=Stealth] (-6.75,6.25) -- (-6,6.25);
\node [font=\LARGE] at (-0.25,9) {$2$};
\node [font=\LARGE] at (-0.25,6.5) {$2$};
\node [font=\LARGE] at (-5.25,9) {$1$};
\node [font=\LARGE] at (-5.25,6.5) {$1$};
\node [font=\LARGE] at (0.75,9) {$U$};
\node [font=\LARGE] at (-4.5,9) {$U$};
\node [font=\LARGE] at (-6.75,7.75) {$y$};
\node [font=\LARGE] at (-5.75,6.25) {$x$};

\node [font=\normalsize] at (1.75,7.5) {$u(y)$};
\node [font=\Large] at (3,7.5) {$\delta$};
\end{circuitikz}
}%

\label{fig:my_label}
\end{figure}

\item A three-phased load is connected to a three-phase balanced supply as shown in the figure. If $v_{an}=100 \angle{0}\degree V,b_{bn}=100\angle{-120}\degree V$ and $V_{cn}=100\angle{-240}\degree V$ (angle are considered positive in the anti-clockwise direction), the value of $R$ for zero current in the neutral wire is $\underline{\hspace{2cm}}\Omega$ (up to $2$ decimal places).\\
\begin{figure}[H]
\centering
\resizebox{0.4\textwidth}{!}{%
\begin{circuitikz}
\tikzstyle{every node}=[font=\LARGE]

\draw (0,23.75) to[short] (1.25,23.75);
\draw (5,20) to[short] (5,18.75);
\draw (0,15) to[short] (1.25,15);
\draw (-3.75,20) to[short] (-3.75,18.75);
\draw (-3.75,19.75) to[short] (-3,19.75);
\draw (-3.75,19) to[short] (-3,19);
\draw (0.25,23.75) to[short] (0.25,23);
\draw (1,23.75) to[short] (1,23);
\draw (5,19.75) to[short] (4.25,19.75);
\draw (5,19) to[short] (4.25,19);
\draw (1,15) to[short] (1,15.75);
\draw (0.25,15) to[short] (0.25,15.75);
\draw (-3,19) to[short] (0.25,15.75);
\draw (1,15.75) to[short] (4.25,19);
\draw (4.25,19.75) to[short] (1,23);
\draw (-3,19.75) to[short] (0.25,23);
\draw (5,19.5) to[short] (5.5,19.5);
\draw (-3.75,19.5) to[short] (-4.25,19.5);
\draw [->, >=Stealth] (-4.25,19.5) -- (-4.25,18.75);
\draw [->, >=Stealth] (5.5,19.5) -- (5.5,18.75);
\draw [->, >=Stealth] (0.5,15) -- (0.5,14.25);
\draw (0.5,23.75) to[sinusoidal voltage source, sources/symbol/rotate=auto] (0.5,25.25);
\node [font=\LARGE] at (-4.25,18.25) {$I$};
\node [font=\LARGE] at (-2,21.5) {$b_1$};
\node [font=\LARGE] at (3,21.5) {$b_2$};
\node [font=\LARGE] at (-1.75,17.25) {$b_3$};
\node [font=\LARGE] at (3.25,17) {$b_4$};
\node [font=\LARGE] at (5.5,18.25) {$I$};
\node [font=\LARGE] at (0.5,13.75) {$I$};
\end{circuitikz}
}%

\label{fig:my_label}
\end{figure}

\item The voltage across the circuit in the figure, and the current through it, are given by the following expressions: \\
$v(t)=5-10 \cos{\omega t+60\degree}V$\\
$i(t)=5+X \cos{\omega t}A$\\
where $\omega=100\pi$ radian/s. If the average power delivered to the circuit is zero, then the value of X (in Ampere) if $\underline{\hspace{2cm}}$  (up to $2$ decimal place).\\
\begin{figure}[!ht]
\centering
\resizebox{0.5\textwidth}{!}{%
\begin{circuitikz}
\tikzstyle{every node}=[font=\Large]
\node [font=\LARGE] at (1,15.25) {};
\begin{scope}[rotate around={-90:(-5.5,15.25)}]
\draw[domain=-5.5:1,samples=100,smooth] plot (\x,{1*sin(1*\x r +5.5 r ) +15.25});
\end{scope}
\begin{scope}[rotate around={85.5:(-5.75,12)}]
\draw[domain=-5.75:-2.5,samples=100,smooth] plot (\x,{1*sin(1*\x r +5.75 r ) +12});
\end{scope}
\draw [short] (-5.5,15.25) -- (1.25,15.25);
\draw [short] (-5.25,8.75) -- (-1.75,8.75);
\draw [short] (-1.75,8.75) -- (-1.75,3.75);
\draw [short] (-1.75,3.75) -- (3.75,3.75);
\draw [short] (3.75,3.75) -- (3.75,10);
\draw [short] (-1.25,8.75) -- (-1.25,4.25);
\draw [short] (-1.25,4.25) -- (3,4.25);
\draw [short] (3,4.25) -- (3,10);
\draw [short] (-1.25,8.75) -- (0.5,8.75);
\draw [short] (0.5,8.75) .. controls (1.75,10) and (1.75,10.25) .. (3,10);
\draw [short] (1.25,15.25) .. controls (5,13) and (5.5,13.5) .. (9,15.25);
\draw [short] (9,15.25) -- (16,15.25);
\draw [short] (3.75,10) .. controls (5.75,10.25) and (6,9.5) .. (6.25,8.75);
\draw [short] (6.25,8.75) -- (16,8.75);
\begin{scope}[rotate around={-92.25:(16,15.25)}]
\draw[domain=16:22.5,samples=100,smooth] plot (\x,{1*sin(1*\x r -16 r ) +15.25});
\end{scope}
\begin{scope}[rotate around={-90:(16.25,12.25)}]
\draw[domain=16.25:19.75,samples=100,smooth] plot (\x,{1*sin(1*\x r -16.25 r ) +12.25});
\end{scope}
\draw [->, >=Stealth] (-2.25,12) -- (2.75,12);
\draw [->, >=Stealth] (7.25,12.25) -- (12.25,12.25);
\draw [->, >=Stealth] (4.75,15.5) -- (4.75,13.75);
\draw [->, >=Stealth] (-2.75,17) -- (-2.75,15.25);
\node [font=\Large] at (1,3) {Mercury Manometer};
\draw (-1.75,5) to[short] (-1.25,5);
\draw (3,6.25) to[short] (3.75,6.25);
\node [font=\Large] at (-2.75,17.75) {diameter $0.25 m$};
\node [font=\Large] at (5,16) {diameter $0.125m$};
\node [font=\Large] at (-3.25,12.75) {$1.$};
\node [font=\Large] at (6.25,12.75) {$2.$};
\end{circuitikz}
}%

\label{fig:my_label}
\end{figure}

\item A phase controlled single rectifier, supplied by an AC source, feeds power to an R-L-E load as shown in the figure. The rectifier output  voltage has an average value gven by$v_0=\frac{v_m}{2\pi}(3+\cos{\alpha})$, where $v_m=80\pi$ volts and $\alpha$ is the firing angle. If the power delivered to the lossless battery is $1600 W, \alpha$ in degree is $\underline{\hspace{2cm}}$  (up to $2$ decimal place).\\
\begin{figure}[H]
\centering
\resizebox{0.4\textwidth}{!}{%
\begin{circuitikz}
\tikzstyle{every node}=[font=\LARGE]
\draw (-2.5,16.25) to[sinusoidal voltage source, sources/symbol/rotate=auto] (-2.5,10);
\draw (-2.5,16.25) to[short] (-1.25,16.25);
\draw (-2.5,10) to[short] (-1.25,10);
\draw (-1.25,17.5) to[short] (1.25,17.5);
\draw (-1.25,17.5) to[short] (-1.25,8.75);
\draw (-1.25,8.75) to[short] (1.25,8.75);
\draw (1.25,8.75) to[short] (1.25,17.5);
\draw (1.25,16.25) to[short, -o] (2.25,16.25) ;
\draw (1.25,10) to[short, -o] (2.25,10) ;
\draw (2.25,16.25) to[short] (3.75,16.25);
\draw (2.25,10) to[short] (3.75,10);
\draw (3.75,16.25) to[R] (3.75,13.75);
\draw (3.75,10) to[battery1] (3.75,11.25);
\draw (3.75,14) to[L ] (3.75,11.25);
\draw (0,12.25) node[ieeestd buffer port, anchor=in, rotate=-270](port){} (port.out) to[short] (0,14.75);
\draw (port.in) to[short] (0,11.25);
\draw [short] (-1,13.5) -- (1,13.5);
\draw (0,13.5) to[short, -o] (0.5,14) ;
\draw [->, >=Stealth] (2.5,11.25) -- (2.5,15.25);
\node [font=\LARGE] at (-4.75,13.25) {$V_m sin(\omega t$)};
\node [font=\LARGE] at (2,13.75) {$v_0$};
\node [font=\LARGE] at (4,10.75) {$+$};
\node [font=\LARGE] at (2.5,9.75) {$-$};
\node [font=\LARGE] at (4.5,15.25) {$2\Omega$};
\node [font=\LARGE] at (5,12.5) {$10mH$};
\node [font=\LARGE] at (4.75,10.75) {$80 V$};
\node [font=\LARGE] at (4.75,10) {$battery$};
\node [font=\LARGE] at (4,10.5) {$-$};
\node [font=\LARGE] at (2.5,16.5) {$+$};
\end{circuitikz}
}%

\label{fig:my_label}
\end{figure}


\item The load resistance is $1 \Omega.$ The capacitor voltage has negligible ripple. Both converters operate in the continuous conduction mode. The switching frequency is $1 kHz$, and the switch control signals arc as shown. The circuit operates in the steady state. Assuming that the converters share the load equally, the average value $i_S1$ , the current of switch $S1$ (in Ampere), is $\underline{\hspace{2cm}}$  (up to $2$ decimal place).\\
\begin{figure}[H]
\centering
\resizebox{0.5\textwidth}{!}{%
\begin{circuitikz}
\tikzstyle{every node}=[font=\LARGE]

\draw [->, >=Stealth] (-6.25,18.75) -- (-4.5,18.75);
\draw  (-4.5,19.5) rectangle (-2.5,18);
\draw [->, >=Stealth] (-2.5,18.75) -- (-0.75,18.75);
\draw  (-0.75,19.5) rectangle (1.5,18);
\draw [->, >=Stealth] (1.5,18.75) -- (3.25,18.75);
\node [font=\LARGE] at (-6.5,19.25) {$Input$};
\node [font=\LARGE] at (-3.75,19) {$\frac{1}{s+1}$};
\node [font=\LARGE] at (0.025,19) {$\frac{s+40}{s+20}$};
\node [font=\LARGE] at (3.25,19.25) {$Output$};
\end{circuitikz}
}%

\label{fig:my_label}
\end{figure}

\item A $3$-phase $900$ kVA, $3$ $kV / \sqrt{3} $k V $(\Delta/Y)$ $50$ Hz transformer has primary (high voltage side) resistance per phase of $0.3 \Omega$ and secondary (low voltage side ) resistance per phase of $0.02 \Omega$. Iron loss of the transformer is $10$ kW. The full load \% efficiency of the transformer operated at unity power factor is $\underline{\hspace{2cm}}$  (up to $2$ decimal place).\\

\item A $200$ V DC series motor, when operating from voltage while driving a certain load, draws $10 A$ current and runs $1000$ r.p.m. The total series resistance is $1 \Omega$. The magnetic circuit is assumed to be linear. At the same voltage, the load torque is increased by $44\%$. The speed of the motor in r.p.m. (rounded to the nearest integer) is $\underline{\hspace{2cm}}$ \\

\item A dc to dc converter shown in the figure is charging a batery bank, B$2$ whose voltage is constant at $150$ V. B$1$ is another battery bank whose voltage is constant at $50$ V. The value of the inductor, L is $5$mH and the ideal switch, S is operated with a switching frequency of $5$ kHz with a duty ratio of $0.4.$ Once the circuit has attained steady state and assuming the diode D to be ideal, The power transferred from B$1$ to B$2$
(in watt) is $\underline{\hspace{2cm}}$  (up to $2$ decimal place).\\
\begin{figure}[H]
\centering
\resizebox{0.4\textwidth}{!}{%
\begin{circuitikz}
\tikzstyle{every node}=[font=\LARGE]
\draw (2.5,13.75) to[battery1] (2.5,8.75);
\draw (2.5,13.75) to[L ] (8.75,13.75);
\draw (9.875,13.75) node[ieeestd buffer port, anchor=in](port){} (port.out) to[short] (12.5,13.75);
\draw (port.in) to[short] (8.75,13.75);
\draw (12.5,13.75) to[battery1] (12.5,8.75);
\draw [->, >=Stealth] (2.5,13.75) -- (3.75,13.75);
\node [font=\LARGE] at (2.75,11.5) {$+$};
\node [font=\LARGE] at (2.75,11) {$-$};
\node [font=\LARGE] at (1.5,10.5) {$50 V$};
\node [font=\LARGE] at (3.5,11.25) {$B1$};
\node [font=\LARGE] at (3.25,14.25) {$i_L$};
\node [font=\LARGE] at (6,14.75) {$L=5 mH$};
\node [font=\LARGE] at (7,11.5) {$S$};
\node [font=\LARGE] at (10.5,14.75) {$D$};
\node [font=\LARGE] at (11.75,11.25) {$B2$};
\node [font=\LARGE] at (13,11.5) {$+$};
\node [font=\LARGE] at (12.75,10.75) {$-$};
\node [font=\LARGE] at (14,11.25) {$150 V$};
\draw (11,14.25) to[short] (11,13.25);
\draw (7.5,13.75) to[short] (7.5,11.75);
\draw (7.5,11.75) to[short] (8.25,11);
\draw (7.5,10.5) to[short] (7.5,8.75);
\draw (2.5,8.75) to[short] (12.5,8.75);
\end{circuitikz}
}%

\label{fig:my_label}
\end{figure}

\item The equivalent circuit of a single phase induction motor is sown in the figure, where the parameters are $R_1=R_2^1=X_{l1}^1=X_{l2}^1=12\Omega, X_M=240\Omega$ and $S$ is the slip. At no-load, the motor speed can be approximated to be the synchronous speed. The no-load lagging power factor of the motor is $\underline{\hspace{2cm}}$  (up to $3$ decimal place).\\
\begin{figure}[H]
\centering
\resizebox{0.4\textwidth}{!}{%
\begin{circuitikz}
\tikzstyle{every node}=[font=\LARGE]
\node at (0,17.5) [circ] {};
\draw (0,17.5) to[R] (2.5,17.5);
\draw (2.5,17.5) to[L ] (5,17.5);
\draw (5,17.5) to[short] (5,16.75);
\draw (4.25,16.75) to[short] (5.75,16.75);
\draw (5.75,16.75) to[R] (5.75,14.25);
\draw (5.75,14.25) to[L ] (5.75,12);
\draw (4.25,16.75) to[L ] (4.25,12);
\draw (4.25,12) to[short] (5.75,12);
\draw (5,12) to[short] (5,11.25);
\draw (4.25,11.25) to[short] (5.75,11.25);
\draw (5.75,11.25) to[R] (5.75,8.75);
\draw (5.75,8.75) to[L ] (5.75,6.5);
\draw (4.25,11.25) to[L ] (4.25,6.5);
\draw (4.25,6.5) to[short] (5.75,6.5);
\draw (5,6.5) to[short] (5,5.5);
\draw (5,5.5) to[short] (0,5.5);
\node at (0,5.5) [circ] {};
\draw [->, >=Stealth] (0,12.25) -- (0,17);
\draw [->, >=Stealth] (0,11.25) -- (0,6.25);
\node [font=\LARGE] at (1.25,11.75) {$V\angle 0\degree$};
\node [font=\LARGE] at (1.25,18.25) {$R_1$};
\node [font=\LARGE] at (3.75,18.5) {$jX_{l1}$};
\node [font=\LARGE] at (3.25,14.5) {$j \frac{X_M}{2}$};
\node [font=\LARGE] at (7.5,13.25) {$j\frac {X_l2^1}{2}$};
\node [font=\LARGE] at (3.25,8.75) {$j \frac{X_M}{2}$};
\node [font=\LARGE] at (7.75,10) {$\frac{R_2^1}{2(2-s)}$};
\node [font=\LARGE] at (7.75,8) {$j \frac{X_{l2}^1}{2}$};
\node [font=\LARGE] at (7.25,15.5) {$\frac{R_2^1}{2s}$};
\end{circuitikz}
}%

\label{fig:my_label}
\end{figure}

\item The voltage $v(t)$ across the terminals $a$ and $b$ as shown in the figure, is a sinusoidal voltage having a frequency $\omega=100$ radian/s. when the inductor current $i(t)$ is in phase with the voltage $v(t)$, the magnitude of the impedance $Z$(in $\Omega$) seen between the terminals $a$ and $b$ is $\underline{\hspace{2cm}}$  (up to $2$ decimal place).\\
\begin{figure}[H]
\centering
\resizebox{0.4\textwidth}{!}{%
\begin{circuitikz}
\tikzstyle{every node}=[font=\LARGE]
\draw (-1.25,17.5) to[L ] (5,17.5);
\draw (5,17.5) to[R] (5,15);
\draw (3.25,17.5) to[C] (3.25,15);
\draw (-1.25,15) to[short] (5,15);
\draw [->, >=Stealth] (1.25,18.25) -- (2.25,18.25);
\draw [->, >=Stealth] (-1.25,16.25) -- (-0.25,16.25);
\node at (-1.25,17.5) [circ] {};
\node at (-1.25,15) [circ] {};
\node at (3.25,15) [circ] {};
\node at (3.25,17.5) [circ] {};
\node [font=\LARGE] at (-2,17.5) {$v(t)$};
\node [font=\LARGE] at (-1.75,15) {$b$};
\node [font=\LARGE] at (0.75,18) {$i(t)$};
\node [font=\LARGE] at (2,15.75) {$100 \mu F$};
\node [font=\LARGE] at (6.25,16) {$100\Omega$};
\node [font=\LARGE] at (-1.25,14.75) {$-$};
\node [font=\LARGE] at (-1.25,17.75) {$+$};
\node [font=\LARGE] at (1.75,16.75) {$L$};
\end{circuitikz}
}%

\label{fig:my_label}
\end{figure}
\end{enumerate}
\end{document}