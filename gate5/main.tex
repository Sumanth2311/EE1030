%iffalse
\let\negmedspace\undefined
\let\negthickspace\undefined
\documentclass[journal,12pt,onecolumn]{IEEEtran}
\usepackage{cite}
\usepackage{amsmath,amssymb,amsfonts,amsthm}
\usepackage{algorithmic}
\usepackage{multicol}
\usepackage{circuitikz}
\usepackage{tikz}
\usepackage{graphicx}
\usepackage{textcomp}
\usepackage{xcolor}
\usepackage{txfonts}
\usepackage{listings}
\usepackage{enumitem}
\usepackage{mathtools}
\usepackage{gensymb}
\usepackage{comment}
\usepackage[breaklinks=true]{hyperref}
\usepackage{tkz-euclide} 
\usepackage{listings}
\usepackage{gvv} 
\usepackage{tikz}
\usetikzlibrary{shapes,arrows} 

%\def\inputGnumericTable{}                                
\usepackage[latin1]{inputenc}                                
\usepackage{color}                                            
\usepackage{array}                                            
\usepackage{longtable}                                       
\usepackage{calc}                                             
\usepackage{multirow}                                         
\usepackage{hhline}                                           
\usepackage{ifthen}                                           
\usepackage{lscape}
\usepackage{tabularx}
\usepackage{array}
\usepackage{float}
\newtheorem{theorem}{Theorem}[section]
\newtheorem{problem}{Problem}
\newtheorem{proposition}{Proposition}[section]
\newtheorem{lemma}{Lemma}[section]
\newtheorem{corollary}[theorem]{Corollary}
\newtheorem{example}{Example}[section]
\newtheorem{definition}[problem]{Definition}
\newcommand{\BEQA}{\begin{eqnarray}}
\newcommand{\EEQA}{\end{eqnarray}}
\newcommand{\define}{\stackrel{\triangle}{=}}
\theoremstyle{remark}


% Marks the beginning of the document
\begin{document}
\bibliographystyle{IEEEtran}
\vspace{3cm}

\title{\textbf{2024-EE-14-26}}
\author{AI24BTEC11027 - R Sumanth}
\maketitle
\bigskip

\renewcommand{\thefigure}{\theenumi}
\renewcommand{\thetable}{\theenumi}
\setlength{\columnsep}{2.5em}
\begin{enumerate}

\item The circuit shown in the figure with the switch $S$ open, is in steady state. After the 
switch $S$ is closed, the time constant of the circuit in seconds is
\begin{figure}[H]
\centering
\resizebox{0.5\textwidth}{!}{%
\begin{circuitikz}
\tikzstyle{every node}=[font=\Large]

\draw (-2.5,21.25) to[normal open switch] (1.25,21.25);
\begin{scope}[rotate around={26.5:(1.25,21.25)}]
\foreach \x in {0,...,-1}{
 
}
\end{scope}
\draw (1.25,21.25) to[R] (3.75,21.25);
\draw (3.75,21.25) to[L ] (6.25,21.25);
\draw (6.25,21.25) to[L ] (6.25,17.5);
\draw (-1.25,17.5) to[L ] (-1.25,19.25);
\draw (-1.25,19.25) to[R] (-1.25,21.25);
\draw (-2.5,17.5) to[american current source] (-2.5,21.25);
\draw (5.75,21.25) to[L ] (5.75,20);
\draw (5.75,20) to[short] (5.75,17.5);
\draw (-2.5,17.5) to[short] (6.25,17.5);
\node [font=\LARGE] at (-3.5,19.5) {1A};
\node [font=\LARGE] at (-0.5,20.25) {1$\Omega$};
\node [font=\LARGE] at (-0.5,18.5) {1H};
\node [font=\LARGE] at (2.5,22) {1$\Omega$};
\node [font=\LARGE] at (-0.5,22) {S};
\node [font=\LARGE] at (4.75,22) {1H};
\node [font=\LARGE] at (5,20.5) {1H};
\node [font=\LARGE] at (7.25,19.5) {1H};
\end{circuitikz}
}%

\label{fig:my_label}
\end{figure}
\begin{enumerate}
    \item $1.25$
    \item $0$
    \item $1$
    \item $1.5$\\
\end{enumerate}

\item Suppose signal $y(t)$ is obtained by the time-reversal of signal $x(t)$, i.e., $y(t)=x(-t),-\infty<t<\infty.$ Which one of the following options is always true for the convolution of $x(t)$ and $y(t)$?
\begin{enumerate}
    \item It is an even signal.
    \item It is an odd signal. 
    \item It is a causal signal.
    \item It is an anti-causal signal. \\
\end{enumerate}

\item If $u(t)$ is the unit step function, then the region of convergence (ROC) of the 
Laplace transform of the signal $x(t)=e^{t^2}[u(t-1)-u(t-10)]$ is
\begin{enumerate}
    \item $-\infty < Re(s) <\infty$
    \item $Re(s)\geq 10$
    \item $Re(s)\leq1$
    \item $1\leq Re(s) \leq10$ \\
\end{enumerate}
\item A three phase, $50 Hz$, $6$ pole induction motor runs at $960 rpm$. The stator copper 
loss, core loss, and the rotational loss of the motor can be neglected. The percentage 
efficiency of the motor is
\begin{enumerate}
    \item $92$
    \item $94$
    \item $96$
    \item $98$\\
\end{enumerate}

\item Which of the following complex functions is/are analytic on the complex plane? 
\begin{enumerate}
    \item $f(z)=jRe(z)$
    \item $Im(z)$
    \item $f(z)=e^{|z|}$
    \item $f(z)=z^2-z$ \\
\end{enumerate}


\item The figure shows the single line diagram of a $4$-bus power network. Branches $b_1, b_2, b_3,$ and $b_4$ have impedances $4_z$,$z$ ,$2_z$, and $4_z$ per-unit (pu), respectively, where  
$z=r+jx$, with $r>0$ and $x>0$. The current drawn from each load bus (marked 
as arrows) is equal to $I$ pu, where $I\neq 0$. If the network is to operate with minimum 
loss, the branch that should be opened is
\begin{figure}[H]
\centering
\resizebox{0.3\textwidth}{!}{%
\begin{circuitikz}
\tikzstyle{every node}=[font=\LARGE]
\draw (-23.5,-37.5) to[short, -o] (-25,-37.5) ;
\draw (-23.75,-35) to[short, -o] (-25,-35) ;
\draw (-23.75,-32.5) to[short, -o] (-25,-32.5) ;
\draw (-23.75,-30) to[short, -o] (-25,-30) ;
\draw (-23.75,-30) to[short] (-17.5,-30);
\draw (-23.5,-37.5) to[short] (-13.75,-37.5);
\draw (-13.75,-37.5) to[short] (-13.75,-36.25);
\draw (-23.75,-35) to[short] (-20,-35);
\draw (-20,-35) to[short] (-20,-36.25);
\draw (-20,-36.25) to[short] (-19,-35.25);
\draw (-18.25,-34.5) to[short] (-17,-33.25);
\draw (-17.25,-33.5) to[short] (-16.75,-33);
\draw (-17.5,-32.5) to[short] (-24,-32.5);
\draw (-16.75,-33) to[short] (-16.75,-31.75);
\draw (-17.5,-30) to[short] (-16.75,-30);
\draw (-16.75,-30) to[short] (-16.75,-30.75);
\draw (-18,-34.25) to[C] (-19.5,-35.75);
\draw (-16.75,-30.5) to[R] (-16.75,-32);
\node at (-16.75,-33) [circ] {};
\draw (-17.25,-32.5) to[short] (-17.75,-32.5);
\draw (-15.75,-34) to[L ] (-14,-35.75);
\draw (-17.25,-32.5) to[short] (-15.5,-34.25);
\draw (-14.25,-35.5) to[short] (-13.75,-36);
\draw (-13.75,-36) to[short] (-13.75,-36.75);
\draw [->, >=Stealth] (-23.75,-30) -- (-22.5,-30);
\draw [->, >=Stealth] (-21.5,-32.5) -- (-22.5,-32.5);
\draw [->, >=Stealth] (-23.5,-35) -- (-22.5,-35);
\draw [->, >=Stealth] (-23.75,-37.5) -- (-22.5,-37.5);
\node [font=\LARGE] at (-25.75,-30) {$a$};
\node [font=\LARGE] at (-25.5,-32.5) {$n$};
\node [font=\LARGE] at (-25.5,-35) {$c$};
\node [font=\LARGE] at (-25.75,-37.5) {$b$};
\node [font=\LARGE] at (-16,-31.5) {$R$};
\node [font=\LARGE] at (-14,-34) {$j10$};
\node [font=\LARGE] at (-17.75,-35.5) {$-j10$};
\end{circuitikz}
}%

\label{fig:my_label}
\end{figure}
\begin{enumerate}
    \item $b_1$
    \item $b_2$
    \item $b_3$
    \item $b_4$ \\
\end{enumerate}

\item For  the block-diagram shown in the figure, the transfer function $\frac{C(s)}{R(s)}$ is
\begin{figure}[H]
\centering
\resizebox{0.2\textwidth}{!}{%
\begin{circuitikz}
\tikzstyle{every node}=[font=\LARGE]
\draw [->, >=Stealth] (0,8.75) -- (2.25,8.75);
\draw  (2.5,7.5) rectangle (5,5);
\draw [->, >=Stealth] (0,3.75) -- (2.25,3.75);
\draw (2.25,8.75) to[short] (3.75,8.75);
\draw (3.75,8.75) to[short] (3.75,7.5);
\draw (2.25,3.75) to[short] (3.75,3.75);
\draw (3.75,3.75) to[short] (3.75,4.75);
\draw (3.75,4.5) to[short] (3.75,5);
\node at (0,3.75) [circ] {};
\node at (0,8.75) [circ] {};
\node [font=\LARGE] at (3.75,6.5) {$Electrical$};
\node [font=\LARGE] at (3.75,5.75) {$Circuit$};
\node [font=\LARGE] at (1.75,8.25) {$+$};
\node [font=\LARGE] at (1.75,3.25) {$-$};
\node [font=\LARGE] at (1.5,9.5) {$i(t)$};
\node [font=\LARGE] at (1.5,6.25) {$v(t)$};
\end{circuitikz}
}%

\label{fig:m}
\end{figure}
\begin{enumerate}
    \item $\frac{G(s)}{1+2G(s)}$
    \item $-\frac{G(s)}{1+2G(s)}$
    \item $\frac{G(s)}{1-2G(s)}$
    \item $-\frac{G(s)}{1-2G(s)}$ \\
\end{enumerate}
\item Consider the standard second-order system of the form $\frac{\omega_n^2}{s^2+2\zeta\omega_n s+\omega_n^2}$ with the 
poles $p$ and $p*$ having negative real parts. The pole locations are also shown in the 
figure. Now consider two such second-order systems as defined below: \\
System $1$: $\omega_n=3 rad/sec and \theta=60\degree$\\
System $2$: $\omega_n=1 rad/sec and \theta=70\degree$\\
\begin{figure}[H]
\centering
\resizebox{0.4\textwidth}{!}{%
\begin{circuitikz}
\tikzstyle{every node}=[font=\LARGE]
\draw (-2.5,16.25) to[sinusoidal voltage source, sources/symbol/rotate=auto] (-2.5,10);
\draw (-2.5,16.25) to[short] (-1.25,16.25);
\draw (-2.5,10) to[short] (-1.25,10);
\draw (-1.25,17.5) to[short] (1.25,17.5);
\draw (-1.25,17.5) to[short] (-1.25,8.75);
\draw (-1.25,8.75) to[short] (1.25,8.75);
\draw (1.25,8.75) to[short] (1.25,17.5);
\draw (1.25,16.25) to[short, -o] (2.25,16.25) ;
\draw (1.25,10) to[short, -o] (2.25,10) ;
\draw (2.25,16.25) to[short] (3.75,16.25);
\draw (2.25,10) to[short] (3.75,10);
\draw (3.75,16.25) to[R] (3.75,13.75);
\draw (3.75,10) to[battery1] (3.75,11.25);
\draw (3.75,14) to[L ] (3.75,11.25);
\draw (0,12.25) node[ieeestd buffer port, anchor=in, rotate=-270](port){} (port.out) to[short] (0,14.75);
\draw (port.in) to[short] (0,11.25);
\draw [short] (-1,13.5) -- (1,13.5);
\draw (0,13.5) to[short, -o] (0.5,14) ;
\draw [->, >=Stealth] (2.5,11.25) -- (2.5,15.25);
\node [font=\LARGE] at (-4.75,13.25) {$V_m sin(\omega t$)};
\node [font=\LARGE] at (2,13.75) {$v_0$};
\node [font=\LARGE] at (4,10.75) {$+$};
\node [font=\LARGE] at (2.5,9.75) {$-$};
\node [font=\LARGE] at (4.5,15.25) {$2\Omega$};
\node [font=\LARGE] at (5,12.5) {$10mH$};
\node [font=\LARGE] at (4.75,10.75) {$80 V$};
\node [font=\LARGE] at (4.75,10) {$battery$};
\node [font=\LARGE] at (4,10.5) {$-$};
\node [font=\LARGE] at (2.5,16.5) {$+$};
\end{circuitikz}
}%

\label{fig:my_label}
\end{figure}

Which one of the following statements is correct?
\begin{enumerate}
    \item Settling time of System $1$ is more than that of System $2$.
    \item Settling time of System $2$ is more than that of System $1$.
    \item Settling times of both the systems are the same. 
    \item Settling time cannot be computed from the given information. \\
\end{enumerate}
\item Consider the cascaded system as shown in the figure. Neglecting the faster 
component of the transient response, which one of the following options is a first
order pole-only approximation such that the steady-state values of the unit step 
responses of the original and the approximated systems are same?  
\begin{figure}[H]
\centering
\resizebox{0.5\textwidth}{!}{%
\begin{circuitikz}
\tikzstyle{every node}=[font=\LARGE]

\draw [->, >=Stealth] (-6.25,18.75) -- (-4.5,18.75);
\draw  (-4.5,19.5) rectangle (-2.5,18);
\draw [->, >=Stealth] (-2.5,18.75) -- (-0.75,18.75);
\draw  (-0.75,19.5) rectangle (1.5,18);
\draw [->, >=Stealth] (1.5,18.75) -- (3.25,18.75);
\node [font=\LARGE] at (-6.5,19.25) {$Input$};
\node [font=\LARGE] at (-3.75,19) {$\frac{1}{s+1}$};
\node [font=\LARGE] at (0.025,19) {$\frac{s+40}{s+20}$};
\node [font=\LARGE] at (3.25,19.25) {$Output$};
\end{circuitikz}
}%

\label{fig:my_label}
\end{figure}
\begin{enumerate}
    \item $\frac{1}{s+1}$
    \item $\frac{2}{s+1}$
    \item $\frac{1}{s+20}$
    \item $\frac{2}{s+20}$\\
\end{enumerate}
\item The table lists two instrument transformers and their features: 
\begin{table}[h!]
\renewcommand{\thetable}{1}
    \centering
   \begin{tabular}{|c| c |  c |}
\hline
\textbf{Variable} & \textbf{Value} & \textbf{Description} \\
\hline
$A$ & \myvec{-1 \\ 2\\ 1\\ \\} & $A$ defined as point \\
\hline
$B$ & \myvec{1 \\ -2\\ 5\\}  & $B$ defined as point  \\
\hline
$C$ & \myvec{4 \\ -7\\ 8\\} & $C$ defined as point\\
\hline
$D$ & \myvec{2 \\ -3\\ 4\\} & $D$ defined as point \\
\hline
\end{tabular} 
\end{table}
\begin{enumerate}
    \item $X$ matches with $P$ and $Q$; $Y$ matches with $R$ and $S$.
    \item $X$ matches with $P$ and $R$; $Y$ matches with $Q$ and $S$.
    \item $X$ matches with $Q$ and $R$; $Y$ matches with $P$ and $S$.
    \item $X$ matches with $Q$ and $S$; $Y$ matches with $P$ and $R$. \\
\end{enumerate}
\item Simplified form of the Boolean function \\ $F(P,Q,R,S)=\Bar{P}\Bar{Q}+\Bar{P}QS+P\Bar{Q}\Bar{R}\Bar{S}+P\Bar{Q}R\Bar{S}$ \\is 
\begin{enumerate}
    \item $\Bar{P}S+\Bar{Q}\Bar{S}$
    \item $\Bar{P}\Bar{Q}+\Bar{Q}\Bar{S}$
    \item $\Bar{P}Q+R\Bar{S}$
    \item $p\Bar{S}+Q\Bar{R}$\\
\end{enumerate}
\item In the circuit, the present value of $Z$ is $1$. Neglecting the delay in the combinatorial 
circuit, the values of $S$ and $Z$, respectively, after the application of the clock will be 
\begin{figure}[H]
\centering
\resizebox{0.4\textwidth}{!}{%
\begin{circuitikz}
\tikzstyle{every node}=[font=\LARGE]
\draw (2.5,13.75) to[battery1] (2.5,8.75);
\draw (2.5,13.75) to[L ] (8.75,13.75);
\draw (9.875,13.75) node[ieeestd buffer port, anchor=in](port){} (port.out) to[short] (12.5,13.75);
\draw (port.in) to[short] (8.75,13.75);
\draw (12.5,13.75) to[battery1] (12.5,8.75);
\draw [->, >=Stealth] (2.5,13.75) -- (3.75,13.75);
\node [font=\LARGE] at (2.75,11.5) {$+$};
\node [font=\LARGE] at (2.75,11) {$-$};
\node [font=\LARGE] at (1.5,10.5) {$50 V$};
\node [font=\LARGE] at (3.5,11.25) {$B1$};
\node [font=\LARGE] at (3.25,14.25) {$i_L$};
\node [font=\LARGE] at (6,14.75) {$L=5 mH$};
\node [font=\LARGE] at (7,11.5) {$S$};
\node [font=\LARGE] at (10.5,14.75) {$D$};
\node [font=\LARGE] at (11.75,11.25) {$B2$};
\node [font=\LARGE] at (13,11.5) {$+$};
\node [font=\LARGE] at (12.75,10.75) {$-$};
\node [font=\LARGE] at (14,11.25) {$150 V$};
\draw (11,14.25) to[short] (11,13.25);
\draw (7.5,13.75) to[short] (7.5,11.75);
\draw (7.5,11.75) to[short] (8.25,11);
\draw (7.5,10.5) to[short] (7.5,8.75);
\draw (2.5,8.75) to[short] (12.5,8.75);
\end{circuitikz}
}%

\label{fig:my_label}
\end{figure}
\begin{enumerate}
    \item $S=0,Z=0$
    \item $S=0,Z=1$
    \item $S=1,Z=0$
    \item $S=1,Z=1$
\end{enumerate}
\item To obtain the Boolean function $F()X,Y=X\Bar{Y}+\Bar{X}$ the inputs $PQRS$ in the figure 
should be
\begin{figure}[H]
\centering
\resizebox{0.4\textwidth}{!}{%
\begin{circuitikz}
\tikzstyle{every node}=[font=\LARGE]
\node at (0,17.5) [circ] {};
\draw (0,17.5) to[R] (2.5,17.5);
\draw (2.5,17.5) to[L ] (5,17.5);
\draw (5,17.5) to[short] (5,16.75);
\draw (4.25,16.75) to[short] (5.75,16.75);
\draw (5.75,16.75) to[R] (5.75,14.25);
\draw (5.75,14.25) to[L ] (5.75,12);
\draw (4.25,16.75) to[L ] (4.25,12);
\draw (4.25,12) to[short] (5.75,12);
\draw (5,12) to[short] (5,11.25);
\draw (4.25,11.25) to[short] (5.75,11.25);
\draw (5.75,11.25) to[R] (5.75,8.75);
\draw (5.75,8.75) to[L ] (5.75,6.5);
\draw (4.25,11.25) to[L ] (4.25,6.5);
\draw (4.25,6.5) to[short] (5.75,6.5);
\draw (5,6.5) to[short] (5,5.5);
\draw (5,5.5) to[short] (0,5.5);
\node at (0,5.5) [circ] {};
\draw [->, >=Stealth] (0,12.25) -- (0,17);
\draw [->, >=Stealth] (0,11.25) -- (0,6.25);
\node [font=\LARGE] at (1.25,11.75) {$V\angle 0\degree$};
\node [font=\LARGE] at (1.25,18.25) {$R_1$};
\node [font=\LARGE] at (3.75,18.5) {$jX_{l1}$};
\node [font=\LARGE] at (3.25,14.5) {$j \frac{X_M}{2}$};
\node [font=\LARGE] at (7.5,13.25) {$j\frac {X_l2^1}{2}$};
\node [font=\LARGE] at (3.25,8.75) {$j \frac{X_M}{2}$};
\node [font=\LARGE] at (7.75,10) {$\frac{R_2^1}{2(2-s)}$};
\node [font=\LARGE] at (7.75,8) {$j \frac{X_{l2}^1}{2}$};
\node [font=\LARGE] at (7.25,15.5) {$\frac{R_2^1}{2s}$};
\end{circuitikz}
}%

\label{fig:my_label}
\end{figure}
\begin{enumerate}
    \item $1010$
    \item $1110$
    \item $0110$
    \item $0001$
\end{enumerate}
























\end{enumerate}
\end{document}
