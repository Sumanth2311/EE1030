%iffalse
\let\negmedspace\undefined
\let\negthickspace\undefined
\documentclass[journal,12pt,onecolumn]{IEEEtran}
\usepackage{cite}
\usepackage{amsmath,amssymb,amsfonts,amsthm}
\usepackage{algorithmic}
\usepackage{multicol}
\usepackage{circuitikz}
\usepackage{tikz}
\usepackage{graphicx}
\usepackage{textcomp}
\usepackage{xcolor}
\usepackage{txfonts}
\usepackage{listings}
\usepackage{enumitem}
\usepackage{mathtools}
\usepackage{gensymb}
\usepackage{comment}
\usepackage[breaklinks=true]{hyperref}
\usepackage{tkz-euclide} 
\usepackage{listings}
\usepackage{gvv} 
\usepackage{tikz}
\usetikzlibrary{shapes,arrows} 

%\def\inputGnumericTable{}                                
\usepackage[latin1]{inputenc}                                
\usepackage{color}                                            
\usepackage{array}                                            
\usepackage{longtable}                                       
\usepackage{calc}                                             
\usepackage{multirow}                                         
\usepackage{hhline}                                           
\usepackage{ifthen}                                           
\usepackage{lscape}
\usepackage{tabularx}
\usepackage{array}
\usepackage{float}
\newtheorem{theorem}{Theorem}[section]
\newtheorem{problem}{Problem}
\newtheorem{proposition}{Proposition}[section]
\newtheorem{lemma}{Lemma}[section]
\newtheorem{corollary}[theorem]{Corollary}
\newtheorem{example}{Example}[section]
\newtheorem{definition}[problem]{Definition}
\newcommand{\BEQA}{\begin{eqnarray}}
\newcommand{\EEQA}{\end{eqnarray}}
\newcommand{\define}{\stackrel{\triangle}{=}}
\theoremstyle{remark}


% Marks the beginning of the document
\begin{document}
\bibliographystyle{IEEEtran}
\vspace{3cm}

\title{\textbf{2014-XE-40-52}}
\author{AI24BTEC11027 - R Sumanth}
\maketitle
\bigskip

\renewcommand{\thefigure}{\theenumi}
\renewcommand{\thetable}{\theenumi}
\setlength{\columnsep}{2.5em}
\begin{enumerate}

\item A fluid is flowing through a pipe of circular cross-section, Reynolds number of the flow is $1600$. the head loss over a $45$ m length of the pipe is $0.6 $m. The average flow velocity of the fluid is $1$ m/s and the acceleration due to gravity is $10 m/s^2$. The diameter of the pipe is $\underline{\hspace{1cm}}$m.\\

\item Consider a laminar flow over a flat plate of width $w$. At section $1-1$, the velocity profile is uniform as shown in the figure. The $x-$ direction velocity profile at section $2-2$ is given by $\frac{u}{U}=2\frac{y}{\delta}-\brak{\frac{y}{\delta}}^2$, where $\delta$ is the boundary layer thickness. 
\begin{figure}[!ht]
\centering
\resizebox{0.5\textwidth}{!}{%
\begin{circuitikz}
\tikzstyle{every node}=[font=\Large]
\draw [->, >=Stealth] (-5,8.75) -- (-3.75,8.75);
\draw [->, >=Stealth] (-5,8.25) -- (-3.75,8.25);
\draw [->, >=Stealth] (-5,7.75) -- (-3.75,7.75);
\draw [->, >=Stealth] (-5,7.25) -- (-3.75,7.25);
\draw [->, >=Stealth] (-5,6.75) -- (-3.75,6.75);
\draw  (-5,8.75) rectangle (-3.75,6.25);
\draw [->, >=Stealth] (-5,6.25) -- (-3.75,6.25);
\draw [->, >=Stealth] (0,8.75) -- (1.25,8.75);
\draw [->, >=Stealth] (0,8.25) -- (1.25,8.25);
\draw [dashed] (-5,6.25) .. controls (-2.75,8.25) and (-1.75,9) .. (2.5,8.75);
\draw [short] (0,6.25) -- (0,8.75);
\draw [->, >=Stealth] (0,6.25) -- (0.25,6.25);
\draw [short] (1.25,8.75) .. controls (1.25,7.5) and (0.75,6.25) .. (0,6.25);
\draw [->, >=Stealth] (0,7.75) -- (1.25,7.75);
\draw [->, >=Stealth] (0,7.25) -- (1,7.25);
\draw [->, >=Stealth] (0,6.75) -- (0.75,6.75);
\draw  (-5,6.25) rectangle (2,6);
\draw [short] (2.25,6.25) -- (2.5,6.25);
\draw [<->, >=Stealth] (2.5,8.75) -- (2.5,6.25);
\draw [->, >=Stealth] (-6.75,6.25) -- (-6.75,7.25);
\draw [->, >=Stealth] (-6.75,6.25) -- (-6,6.25);
\node [font=\LARGE] at (-0.25,9) {$2$};
\node [font=\LARGE] at (-0.25,6.5) {$2$};
\node [font=\LARGE] at (-5.25,9) {$1$};
\node [font=\LARGE] at (-5.25,6.5) {$1$};
\node [font=\LARGE] at (0.75,9) {$U$};
\node [font=\LARGE] at (-4.5,9) {$U$};
\node [font=\LARGE] at (-6.75,7.75) {$y$};
\node [font=\LARGE] at (-5.75,6.25) {$x$};

\node [font=\normalsize] at (1.75,7.5) {$u(y)$};
\node [font=\Large] at (3,7.5) {$\delta$};
\end{circuitikz}
}%

\label{fig:my_label}
\end{figure}
The volume flow rate through section $2-2$ is given by 
\begin{enumerate}
    \item $\frac{1}{2}Uw\delta$
    \item $\frac{1}{3}Uw\delta$
    \item $Uw\delta$
    \item $\frac{2}{3}Uw\delta$
\end{enumerate}
\item A cube of weight $W$ and side a falls at a constant speed in a medium as shown in the figure. If the medium is air (mass density = $\rho$ air) let $U$air be the velocity of the cube . If the medium is water (mass density = $\rho$ water) let $U$ water be the velocity od the cube. 
\begin{figure}[H]
\centering
\resizebox{0.4\textwidth}{!}{%
\begin{circuitikz}
\tikzstyle{every node}=[font=\LARGE]

\draw (0,23.75) to[short] (1.25,23.75);
\draw (5,20) to[short] (5,18.75);
\draw (0,15) to[short] (1.25,15);
\draw (-3.75,20) to[short] (-3.75,18.75);
\draw (-3.75,19.75) to[short] (-3,19.75);
\draw (-3.75,19) to[short] (-3,19);
\draw (0.25,23.75) to[short] (0.25,23);
\draw (1,23.75) to[short] (1,23);
\draw (5,19.75) to[short] (4.25,19.75);
\draw (5,19) to[short] (4.25,19);
\draw (1,15) to[short] (1,15.75);
\draw (0.25,15) to[short] (0.25,15.75);
\draw (-3,19) to[short] (0.25,15.75);
\draw (1,15.75) to[short] (4.25,19);
\draw (4.25,19.75) to[short] (1,23);
\draw (-3,19.75) to[short] (0.25,23);
\draw (5,19.5) to[short] (5.5,19.5);
\draw (-3.75,19.5) to[short] (-4.25,19.5);
\draw [->, >=Stealth] (-4.25,19.5) -- (-4.25,18.75);
\draw [->, >=Stealth] (5.5,19.5) -- (5.5,18.75);
\draw [->, >=Stealth] (0.5,15) -- (0.5,14.25);
\draw (0.5,23.75) to[sinusoidal voltage source, sources/symbol/rotate=auto] (0.5,25.25);
\node [font=\LARGE] at (-4.25,18.25) {$I$};
\node [font=\LARGE] at (-2,21.5) {$b_1$};
\node [font=\LARGE] at (3,21.5) {$b_2$};
\node [font=\LARGE] at (-1.75,17.25) {$b_3$};
\node [font=\LARGE] at (3.25,17) {$b_4$};
\node [font=\LARGE] at (5.5,18.25) {$I$};
\node [font=\LARGE] at (0.5,13.75) {$I$};
\end{circuitikz}
}%

\label{fig:my_label}
\end{figure}
Neglecting the buoyancy force and assuming drag coefficient to be same for both cases, the ratio of velocities, $\frac{U_air}{U_water}$ is given by 
\begin{enumerate}
    \item $\frac{\rho_{air}}{\rho_{water}}$
    \item $\sqrt{\frac{\rho_{air}}{\rho_{water}}}$
    \item $1$
    \item $\sqrt{\frac{\rho_{water}}{\rho_{air}}}$\\
\end{enumerate}
\item Water is flowing through a venturimeter having a diameter of $0.25$ m at the entrance (station $1$) and $0.125$m at the throat (station $2$) as shown in the figure. A mercury manometer measure the piezometric head difference between stations $1$ and $2$ as $1.3505$m. The loss of head between these two stations, is $\frac{1}{7}$ times the velocity head at the station $2$. Assume the acceleration due to gravity to be $10 m/s^2$. The velocity of water at the throat is $\underline{\hspace{1cm}} m/s.$\\
\begin{figure}[!ht]
\centering
\resizebox{0.5\textwidth}{!}{%
\begin{circuitikz}
\tikzstyle{every node}=[font=\Large]
\node [font=\LARGE] at (1,15.25) {};
\begin{scope}[rotate around={-90:(-5.5,15.25)}]
\draw[domain=-5.5:1,samples=100,smooth] plot (\x,{1*sin(1*\x r +5.5 r ) +15.25});
\end{scope}
\begin{scope}[rotate around={85.5:(-5.75,12)}]
\draw[domain=-5.75:-2.5,samples=100,smooth] plot (\x,{1*sin(1*\x r +5.75 r ) +12});
\end{scope}
\draw [short] (-5.5,15.25) -- (1.25,15.25);
\draw [short] (-5.25,8.75) -- (-1.75,8.75);
\draw [short] (-1.75,8.75) -- (-1.75,3.75);
\draw [short] (-1.75,3.75) -- (3.75,3.75);
\draw [short] (3.75,3.75) -- (3.75,10);
\draw [short] (-1.25,8.75) -- (-1.25,4.25);
\draw [short] (-1.25,4.25) -- (3,4.25);
\draw [short] (3,4.25) -- (3,10);
\draw [short] (-1.25,8.75) -- (0.5,8.75);
\draw [short] (0.5,8.75) .. controls (1.75,10) and (1.75,10.25) .. (3,10);
\draw [short] (1.25,15.25) .. controls (5,13) and (5.5,13.5) .. (9,15.25);
\draw [short] (9,15.25) -- (16,15.25);
\draw [short] (3.75,10) .. controls (5.75,10.25) and (6,9.5) .. (6.25,8.75);
\draw [short] (6.25,8.75) -- (16,8.75);
\begin{scope}[rotate around={-92.25:(16,15.25)}]
\draw[domain=16:22.5,samples=100,smooth] plot (\x,{1*sin(1*\x r -16 r ) +15.25});
\end{scope}
\begin{scope}[rotate around={-90:(16.25,12.25)}]
\draw[domain=16.25:19.75,samples=100,smooth] plot (\x,{1*sin(1*\x r -16.25 r ) +12.25});
\end{scope}
\draw [->, >=Stealth] (-2.25,12) -- (2.75,12);
\draw [->, >=Stealth] (7.25,12.25) -- (12.25,12.25);
\draw [->, >=Stealth] (4.75,15.5) -- (4.75,13.75);
\draw [->, >=Stealth] (-2.75,17) -- (-2.75,15.25);
\node [font=\Large] at (1,3) {Mercury Manometer};
\draw (-1.75,5) to[short] (-1.25,5);
\draw (3,6.25) to[short] (3.75,6.25);
\node [font=\Large] at (-2.75,17.75) {diameter $0.25 m$};
\node [font=\Large] at (5,16) {diameter $0.125m$};
\node [font=\Large] at (-3.25,12.75) {$1.$};
\node [font=\Large] at (6.25,12.75) {$2.$};
\end{circuitikz}
}%

\label{fig:my_label}
\end{figure}

\item Neoprene is rendered non-inflammable because 
\begin{enumerate}
    \item it has a highly cross-linked structure 
    \item it has a highly linear chain structure 
    \item of the presence of chlorine atom in the structure 
    \item of the absence of chlorine atom in the structure \\
\end{enumerate}
\item Nylone-$6$ is manufactured from 
\begin{enumerate}
    \item caprolactum 
    \item adipic acid ad hexamethylene diamine 
    \item malcic anhydride and hexamethylene diamine 
    \item secasic acid and hexamethylene diamine\\
\end{enumerate}
\item At room temperature, the typical barrier potential for silicon p-n junction in Volt(V) is 
\begin{enumerate}
    \item $0.7\times 10^{-23}$
    \item $0.07$
    \item $0.70$
    \item $7.0$\\
\end{enumerate}
\item Quantitative measurements of the roughness of a polysilicon wafer can be performed with 
\begin{enumerate}
    \item scanning tunneling microscopy 
    \item scanning electron microscopy
    \item transmission electron microscopy 
    \item atomic force microscopy\\
\end{enumerate}
\item The temperature of the antiferromagnetic-to-paramagnetic transition is called 
\begin{enumerate}
    \item Curic temperature
    \item Curie-weiss temperature 
    \item Neel temperature
    \item Debye temperature \\
\end{enumerate}
\item At low injection level, a forward biased p-n junction would have 
\begin{enumerate}
    \item no charge carriers
    \item minority carrier concentration much more than majority carrier concentration
    \item minority carrier concentration  equal to majority carrier concentration
    \item minority carrier concentration much less than majority carrier concentration\\
\end{enumerate}
\item Which of the following mechanical properties of a material depend on the mobile dislocation density in it.\\
(P) Young's modulus (Q) yield strength (R) ductility (S) fracture toughness
\begin{enumerate}
    \item $P,Q,R$
    \item $Q,R,S$
    \item $P,R,S$
    \item $S,P,Q$
\end{enumerate}
\item The equilibrium concentration of vacancies in a pure metal 
\begin{enumerate}
    \item increase exponentially with temperature 
    \item decrease exponential with temperature 
    \item varies linearly with temperature 
    \item is independent of temperature 
\end{enumerate}
\item The materials belonging to which one of the following crystal classes would be both piezoelectric
\begin{enumerate}
    \item $222$
    \item $4mm$
    \item $\Bar{1}$
    \item $2/m$
    
\end{enumerate}
\end{enumerate}
\end{document}